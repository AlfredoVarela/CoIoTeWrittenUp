\cleardoublepage
\phantomsection
\chapter*{Introducció}
En los últimos años, el avance sobre los temas de IoT han llevado a crecimiento muy elevado en la investigación tanto como dispositivos como de tecnologías para el cumplimiento de todas las barreras que nos ponían el software y el hardware que disponíamos hasta conseguir protocolos muy simples y sensores extremadamente pequeños y baratos. La demanda de este tipo de productos hace que muchas empresas cambie su orientación de mercado y se centren en el mundo de los sensores o incluso invierten en hacer una instalación de este tipo para mejorar su productividad. 

Por eso se decidió realizar este proyecto, el cual se llevará a cabo el despliegue de la plataforma aaaida en una Raspberry. La cual nos puede permitir el uso de dicha plataforma y sus aplicaciones en lugares a los cuales la conexión a internet es nula o escasa, hayan sufrido un desastre natural... ya que es una plataforma centrada en Ihealth que nos permite la monitorización del estado de un paciente.
Gracias a las conexiones de la Raspberry podremos realizar una red con sensores que mediante Bluetooth se podrán comunicar y almacenar los datos. 

Por otra parte el despliegue de la aplicación se llevará a cabo usando Docker, un software que permite contenerizar todo aquello que nuestra aplicaciones necesiten. El cual no sera muy util ya que nos permite instalar o actualizar contenedores por separado y no todo el sistema cada vez. Docker es utilizado básicamente en el mundo de servidores y no de IoT lo cual las ventajas que nos proporciona a la hora del despliegue son contrarrestadas con su complejidad en sistemas con capacidades limitadas.

Nuestra intención es poder desplegar toda la infraestructura de la empresa utilizando un sistema nuevo para IoT, comprobar su viabilidad y beneficios, ofreciendo un producto el cual podría ser de gran utilidad en caso de desastres como seria un sistemas de monitorización de paciente. 

La utilización de la Raspberry hace que esta infraestructura no sea extrapolable al cien por cien de los despliegues de IoT ya que sus especificaciones son bastantes superiores a las de un sensores utilizados normalmente pero como aproximación y prueba piloto para  las tecnologías nombradas anteriormente podrían abrirse un hueco en el mundo es suficiente. 

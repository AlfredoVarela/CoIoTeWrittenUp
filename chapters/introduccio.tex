\cleardoublepage
\phantomsection
\chapter*{Introducción}

En los últimos años, las investigaciones sobre los temas de IoT han avanzado notablemente tanto en dispositivos como en tecnologías. Con el objetivo de superar las barreras impuestas por el software y el hardware disponibles, se han conseguido protocolos muy simples y sensores extremadamente pequeños y baratos. La demanda de este tipo de productos hace que muchas empresas cambien su orientación de mercado y se centren en el mundo de los sensores, incluso invierten en hacer una instalación de este tipo para mejorar su productividad. 

Por eso se decidió realizar este proyecto, en el que se llevará a cabo el despliegue de la plataforma Aaaida en una Raspberry Pi. Ésta nos puede permitir el uso de dicha plataforma y sus aplicaciones en lugares en los cuales la conexión a internet es nula, en lugares que hayan sufrido un desastre natural, etc. Aaaida es una plataforma centrada en eHealth, que nos permite la monitorización del estado de un paciente, cosa que afirma la necesidad de poder utilizarla en los casos de emergencia nombrados anteriormente. Gracias a las conexiones de la Raspberry Pi, se puede realizar una red con sensores que mediante Bluetooth se puedan comunicar y puedan almacenar los datos. 

Por otra parte, el despliegue de la aplicación se llevará a cabo usando Docker, un software que permite añadir en contenedores todo aquello que nuestras aplicaciones necesiten. Éste será muy útil, ya que nos permite instalar o actualizar contenedores por separado y no tener que hacerlo para todo el sistema. Docker es utilizado básicamente en el mundo de servidores y no en IoT. Las ventajas que nos proporcionará a la hora del despliegue son muchas pero hay que tener en cuenta su complejidad en sistemas con capacidades limitadas.

Nuestra intención, es poder desplegar toda la infraestructura de la empresa utilizando un sistema nuevo para IoT y poder comprobar su viabilidad y sus beneficios. Se llegará a ofrecer un producto que podría ser de gran utilidad en caso de desastres, como sería un sistema de monitorización de pacientes mediante sensores. 

La utilización de la Raspberry Pi, que es un dispositivo muy utilizado en IoT por sus grandes prestaciones a bajo precio, puede servir como aproximación y prueba piloto para probar la viabilidad de la tecnología nombrada anteriormente. Docker podría abrirse un hueco en el mundo de IoT, aunque originalmente está orientado a arquitecturas de servidor con recursos ilimitados, pero puede ser utilizado perfectamente con los recursos que nos ofrece una Raspberry Pi. 

El sensor utilizado para realizar este proyecto será una banda HRM (Heart Rate Monitor), más concretamente el Zephyr BioHarness 3, un dispositivo con conexión bluetooth con el que se podrá realizar la red de comunicación y generar los datos para gestionarlos en la plataforma de Aaaida. 
\chapter*{Glosario}

{\bf IoT}: Internet Of Things

{\bf Journalig}: Memoria caché en la cual se tiene constancia de las escrituras hechas en disco. Esto en caso de fallo o desconexión no prevista ayuda a evitar la corrupción de datos. Por esta razón se activa, ya que la Raspberry Pi es un dispositivo sin batería y si hay un corte de luz, puede que se corrompa la base de datos y el restar no se pueda ejecutar. 

{\bf Daemon}: Proceso en segundo plano que se inicia como servicio.

{\bf API}: Application Programming Interface. Es un conjunto de especificaciones
para la comunicación entre diferentes componentes software.

{\bf CRC}: Código de detección de errores.

{\bf Payload}: Los datos transmitidos en una comunicación.

{\bf JSON}: Javascript Object Notation.

{\bf MVC}: El modelo vista controlador, es un patrón de arquitectura de software que separa los
datos y la lógica de una interfaz de usuario y el módulo encargado de gestionar cada uno de sus
eventos y comunicaciones.

{\bf SPA}: Single-page Application, es una aplicación web que cabe en una sola página con el
objetivo de proporcionar una experiencia fluida a los usuarios.


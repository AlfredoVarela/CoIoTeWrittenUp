\chapter{Visión General del proyecto}\label{C:Visión General del proyecto}

El propósito de este proyecto es desplegar toda la infraestructura de la empresa AlterAid en un dispositivo móvil en nuestro caso una Raspberry Pi. Con la finalidad de poder conseguir desplegar nuestras aplicaciones en lugares remotos o sin conexión a internet o que han sufrido un desastre natural.

Como segundo objetivo del proyecto queremos implementar un pequeño despliegue sensores haciendo que nuestra Raspberry sea el nodo central o sumidero de datos. 
Para poder llevar a cabo todo esto es necesario contar con la utilización de los contenedores usando la tecnología Docker. Que nos facilitaran el trabajo y es motivo de investigación su utilización fuera del mundo de servidores.

\section{Docker}

Es una plataforma desarrollada por la empresa AlterAid la cual nos permite crearnos un usuario y con este usuario poder tener varios vínculos. Un Vínculo es un ente que representa aquello por el que el Usuario se ocupa y preocupa. Ejemplos de Vínculos de Usuario pueden ser familiares, pacientes, amigos, etc. 
Esta es una plataforma web la cual recoge todos los datos de las aplicaciones de la empresa y los almacena. Por eso la importancia de poder desplegar toda la infraestructura en un dispositivo que sea fácil de transportar y poder llegar a cualquier lugar.

\section{Aaaida}

La presentació del document ha de ser a dues cares a partir de la Introducció i fins al final del document.


\section{Marges}

Pel que fa als marges, no cal fer absolutament res. Aquesta plantilla \LaTeX \ ja ho fa per vosaltres :-)


\section{Tipografia}

\subsection{Tipus de lletra}

Pel que fa al tipus de lletra, no cal fer absolutament res. Aquesta plantilla \LaTeX \ ja ho fa per vosaltres :-)

A part dels capítols, només podeu utilitzar tres nivells de profunditat en la divisió per apartats:

\begin{verbatim}
\chapter{Nom del capítol}
\section{Nom de l'apartat}
\subsection{Nom del sub-apartat}
\subsubsection{Nom del sub-sub-apartat}
\end{verbatim}



\subsection{Interlineat}

Pel que fa al interlineat, ja sigui entre línies o finals de paràgrafs/seccions/capitols, no cal fer absolutament res. Aquesta plantilla \LaTeX \ ja ho fa per vosaltres :-)


\section{Numeració dels títols}

Pel que fa a la numeració dels capítols, apartats, subapartats i subsubapartats, no cal fer absolutament res. Aquesta plantilla \LaTeX \ ja ho fa per vosaltres :-)


\section{Encapçalaments i números de pàgina}

Pel que fa als encapçalaments i números de pàgina, no cal fer absolutament res. Aquesta plantilla \LaTeX \ ja ho fa per vosaltres :-)


\section{Enquadernació}

Cal realitzar l'enquadernació amb espiral negra, i tapes de plàstic transparent. La portada ha de ser de cartolina de color blau (color Pantone 542 o el més similar possible).

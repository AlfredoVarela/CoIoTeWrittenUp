%%%%%%%%%%%%%%%%%%%%%%%%%%%%%%%%%%%%%%%%%%%%%%%%%%%%%%%%%%%%%%%%%%%%%%%%%%%%%
%%%%%%                                                                  %%%%% 
%%%%%%          Maqueta de memòria TFC/PFC de l'EETAC                   %%%%% 
%%%%%%                                                                  %%%%% 
%%%%%%%%%%%%%%%%%%%%%%%%%%%%%%%%%%%%%%%%%%%%%%%%%%%%%%%%%%%%%%%%%%%%%%%%%%%%%
%%%%%%%%%%%%%%%%%%%%%%%%%%%%%%%%%%%%%%%%%%%%%%%%%%%%%%%%%%%%%%%%%%%%%%%%%%%%%
%%                                                                         %%
%%          Autor: Xavier Prats i Menéndez (xavier.prats@upc.edu)          %% 
%%                  Technical University of Catalonia (UPC)                %%
%%                                                                         %%
%%%%%%%%%%%%%%%%%%%%%%%%%%%%%%%%%%%%%%%%%%%%%%%%%%%%%%%%%%%%%%%%%%%%%%%%%%%%%
%%      This work is licensed under the Creative Commons  Attribution-     %%
%%   -Noncommercial-Share Alike 3.0 Spain License. To view a copy of this  %% 
%%    license, visit http://creativecommons.org/licenses/by-nc-sa/3.0/es/  %%
%%    or send a letter to Creative Commons, 171 Second Street, Suite 300,  %%
%%                  San Francisco,California, 94105, USA.                  %%
%%%%%%%%%%%%%%%%%%%%%%%%%%%%%%%%%%%%%%%%%%%%%%%%%%%%%%%%%%%%%%%%%%%%%%%%%%%%%
%% Versió 2.1 - Juliol 2012                                                %%
%%%%%%%%%%%%%%%%%%%%%%%%%%%%%%%%%%%%%%%%%%%%%%%%%%%%%%%%%%%%%%%%%%%%%%%%%%%%%

%%% NOTA: els seguents packages son necessaris per utilitzar la
%%%       plantilla seguent:
%%%       ifthen,calc,helvet,pslatex,fancyhdr,nextpage,subfigure,tocloft,graphicx,url

%%% NOTA: Es possible que algunes distribuicions Linux o Windows.
%%%       no portin aquests paquets instal·lats per defecte.
%%%       En aquest cas els haureu d'instal·lar manualment.


%%%%%%%%%%%%%%%%%%%%%%%%%%%%%%%%%%%%%%%%%%%%%%%%%%%%%%%%%%%%%%%%%%%%%%%%%%%%%
% 1- INICIALITZACIÓ
%%%%%%%%%%%%%%%%%%%%%%%%%%%%%%%%%%%%%%%%%%%%%%%%%%%%%%%%%%%%%%%%%%%%%%%%%%%%%

\documentclass[spanish,final]{setup/eetac_tfc_pfc}
%% * OPCIONS A CONFIGURAR al \documentclass
%%    - Estat del document: final o draft
%%      NOTA: Draft no inserta les figures i marca només l'espai que
%%      ocupen. També s'indica quan el text sobrepassa els marges.
%%      Draft és molt útil per compilar ràpid el document si no és important
%%      en aquell moment visualitzar les figures.
%%    - Idioma PRINCIPAL del document: catalan, spanish, english, french...

\usepackage[english,catalan, spanish]{babel}
%%  * INCLOURE TOTS ELS IDIOMES QUE S'USARAN EN EL DOCUMENT
%%    NOTA: per canviar d'idioma al mig del document usar:
%%          \selectlanguage{nom_idioma}
%%%%%%%%%%%%%%%%%%%%%%%%%%%%%%%%%%%%%%%%%%%%%%%%%%%%%%%%%%%%%%%%%%%%%%%%%%%%%

%%%%%%%%%%%%%%%%%%%%%%%%%%%%%%%%%%%%%%%%%%%%%%%%%%%%%%%%%%%%%%%%%%%%%%%%%%%%%
% 2- CÀRREGA DE PAQUETS ADICIONALS (OPCIONALS)
%%%%%%%%%%%%%%%%%%%%%%%%%%%%%%%%%%%%%%%%%%%%%%%%%%%%%%%%%%%%%%%%%%%%%%%%%%%%%

%%% NOTA: Es possible que algunes distribuicions Linux o Windows.
%%%       no portin aquests paquets instal·lats per defecte.
%%%       En aquest cas els haureu d'instal·lar manualment.

%% El paquet inputenc és extramadament útil. 
%% Permet escriure els accents directament amb l'editor de texte
%% sense haver de fer coses com per exemple: introducci\'o
%% Heu d'especificar la codificació de caracters que utilitzeu pel
%% vostre fitxer (en aquest exemple utf8)
\usepackage[utf8]{inputenc}


%Define the listing package
\usepackage{listings} %code highlighter
\usepackage{color} %use color
\definecolor{mygreen}{rgb}{0,0.6,0}
\definecolor{mygray}{rgb}{0.5,0.5,0.5}
\definecolor{mymauve}{rgb}{0.58,0,0.82}
 
%Customize a bit the look
\lstset{ %
backgroundcolor=\color{white}, % choose the background color; you must add \usepackage{color} or \usepackage{xcolor}
basicstyle=\footnotesize, % the size of the fonts that are used for the code
breakatwhitespace=false, % sets if automatic breaks should only happen at whitespace
breaklines=true, % sets automatic line breaking
captionpos=b, % sets the caption-position to bottom
commentstyle=\color{mygreen}, % comment style
deletekeywords={...}, % if you want to delete keywords from the given language
escapeinside={\%*}{*)}, % if you want to add LaTeX within your code
extendedchars=true, % lets you use non-ASCII characters; for 8-bits encodings only, does not work with UTF-8
frame=single, % adds a frame around the code
keepspaces=true, % keeps spaces in text, useful for keeping indentation of code (possibly needs columns=flexible)
keywordstyle=\color{blue}, % keyword style
% language=Octave, % the language of the code
morekeywords={*,...}, % if you want to add more keywords to the set
%numbers=left, % where to put the line-numbers; possible values are (none, left, right)
%numbersep=5pt, % how far the line-numbers are from the code
%numberstyle=\tiny\color{mygray}, % the style that is used for the line-numbers
rulecolor=\color{black}, % if not set, the frame-color may be changed on line-breaks within not-black text (e.g. comments (green here))
showspaces=false, % show spaces everywhere adding particular underscores; it overrides 'showstringspaces'
showstringspaces=false, % underline spaces within strings only
showtabs=false, % show tabs within strings adding particular underscores
stepnumber=1, % the step between two line-numbers. If it's 1, each line will be numbered
stringstyle=\color{mymauve}, % string literal style
tabsize=2, % sets default tabsize to 2 spaces
title=\lstname % show the filename of files included with \lstinputlisting; also try caption instead of title
}
%END of listing package%
 
\definecolor{darkgray}{rgb}{.4,.4,.4}
\definecolor{purple}{rgb}{0.65, 0.12, 0.82}
 
%define Javascript language
\lstdefinelanguage{JavaScript}{
keywords={typeof, new, true, false, catch, function, return, null, catch, switch, var, if, in, while, do, else, case, break},
keywordstyle=\color{blue}\bfseries,
ndkeywords={class, export, boolean, throw, implements, import, this},
ndkeywordstyle=\color{darkgray}\bfseries,
identifierstyle=\color{black},
sensitive=false,
comment=[l]{//},
morecomment=[s]{/*}{*/},
commentstyle=\color{purple}\ttfamily,
stringstyle=\color{red}\ttfamily,
morestring=[b]',
morestring=[b]"
}
 
\lstset{
language=JavaScript,
extendedchars=true,
basicstyle=\footnotesize\ttfamily,
showstringspaces=false,
showspaces=false,
%numbers=left,
%numberstyle=\footnotesize,
%numbersep=9pt,
tabsize=2,
breaklines=true,
showtabs=false,
captionpos=b
}


\usepackage{listingsutf8}

\definecolor{olive}{rgb}{0.5, 0.5, 0.0}
\lstdefinelanguage{docker}{
  keywords={FROM, RUN, COPY, ADD, ENTRYPOINT, CMD,  ENV, WORKDIR, EXPOSE, LABEL, USER, VOLUME, STOPSIGNAL, ONBUILD, MAINTAINER},
  keywordstyle=\color{blue}\bfseries,
  identifierstyle=\color{black},
  sensitive=false,
  comment=[l]{\#},
  commentstyle=\color{purple}\ttfamily,
  stringstyle=\color{red}\ttfamily,
  morestring=[b]',
  morestring=[b]"
}

\lstdefinelanguage{docker-compose}{
  keywords={image, environment, ports, container_name, ports, volumes, links},
  keywordstyle=\color{blue}\bfseries,
  identifierstyle=\color{black},
  sensitive=false,
  comment=[l]{\#},
  commentstyle=\color{purple}\ttfamily,
  stringstyle=\color{red}\ttfamily,
  morestring=[b]',
  morestring=[b]"
}
\lstdefinelanguage{docker-compose-2}{
  keywords={version, volumes, services},
  keywordstyle=\color{blue}\bfseries,
  keywords=[2]{image, environment, ports, container_name, ports, links, build},
  keywordstyle=[2]\color{olive}\bfseries,
  identifierstyle=\color{black},
  sensitive=false,
  comment=[l]{\#},
  commentstyle=\color{purple}\ttfamily,
  stringstyle=\color{red}\ttfamily,
  morestring=[b]',
  morestring=[b]"
}

\lstset{basicstyle=\ttfamily,
  showstringspaces=false,
  commentstyle=\color{red},
  keywordstyle=\color{blue},
  inputencoding=utf8,
  extendedchars=true
}

\usepackage{xcolor}
\usepackage{listings}
\lstdefinestyle{Bash}
{language=bash,
keywordstyle=\color{blue},
basicstyle=\ttfamily,
morekeywords={peter@kbpet},
%alsoletter={:~$},
morekeywords=[2]{peter@kbpet:},
keywordstyle=[2]{\color{red}},
literate={\$}{{\textcolor{red}{\$}}}1 
         {:}{{\textcolor{red}{:}}}1
         {~}{{\textcolor{red}{\textasciitilde}}}1,
}

%% Símbols matemàtics de la American Mathematical Society
\usepackage{amssymb,amsmath,amsfonts}  

%% El paquet array proporciona eines molt útils a l'hora de fer 
%% equacions amb matrius
\usepackage{array}             

%% Paquet que permet fer taules fusionant cel·les de files consecutives
\usepackage{multirow}          

%% Paquet molt útil en cas de tenir taules molt llargues que 
%   ocupin vàries pàgines
\usepackage{longtable}          

%% Permet canviar els colors del document
%\usepackage{color,colortbl}

%% Paquet molt útil que permet activar links en el PDF final.
%% * NO OBLIDAR DE CONFIGURAR els quatre primer camps!
\usepackage[
  pdfauthor={Nom desplegado yCognoms autor},            % Configurar adientment
  %pdftitle={Treball Fi de Carrera - autor}, % Configurar adientment
  %pdfsubject={Titol del TFC aqui},          % Configurar adientment
  % Modificació respecte a la versió 2.1 - Iván Padilla Montero - Juliol 2014
  pdftitle={Treball Fi de Grau - autor}, % Configurar adientment
  pdfsubject={Titol del TFG aqui},          % Configurar adientment  
  pdfkeywords={keyword1, keyword2, ...},    % Configurar adientment
  pdfcreator={EETAC-UPC}, 
  pdfproducer={LaTeX, dvipdf},
  pdfdisplaydoctitle=true, plainpages=false, linktocpage=true,         
  colorlinks=true, linkcolor=blue,citecolor=blue,urlcolor=blue,
  hyperfootnotes=false, pagebackref=true, pdfpagelabels=true,
  pdfpagemode=UseOutlines,
]{hyperref} 

%% NOTA IMPORTANT!:
%% Per tal que hyperef funcioni correctament amb els capitols o seccions no
%% numerats (\chapter*{}), com per exemple introducció, conclusions i bibliografia
%% cal posar les dues comandes seguents ABANS del \chapter*{} en questió
%\cleardoublepage
%\phantomsection

%% Permet trencar links URL. 
%% Atenció! afegir aquest paquet DESPRES del hyperref!!
\usepackage{breakurl} 

%% Permet arranjar matricialment multiples figures
%% NOTA: afegir aquest paquet DESPRES del hyperref!!
%%       Si no es desitja utilitzar aquest paquet, comentar la linia seguent
%%       i anar TAMBE al fitxer de classe (eetac_tfc_pfc.cls) per substituir: 
%%       \RequirePackage[subfigure]{tocloft}  per  \RequirePackage{tocloft}
\usepackage{subfigmat}         
\usepackage{listings}
%%%%%%%%%%%%%%%%%%%%%%%%%%%%%%%%%%%%%%%%%%%%%%%%%%%%%%%%%%%%%%%%%%%%%%%%%%%%%


%%%%%%%%%%%%%%%%%%%%%%%%%%%%%%%%%%%%%%%%%%%%%%%%%%%%%%%%%%%%%%%%%%%%%%%%%%%%%
% 3- DOCUMENT
%%%%%%%%%%%%%%%%%%%%%%%%%%%%%%%%%%%%%%%%%%%%%%%%%%%%%%%%%%%%%%%%%%%%%%%%%%%%%

%%% Configuració de les dades i variables boleanes rellevants del document:
\input{setup/dades.tex}  

%%% Configuració de MACROS o ENTORNS (opcionals) definides per l'usuari:
\input{setup/user-macros.tex}  

%%% Configuració manual de les regles d'hyphenation:
\input{setup/hyphenation.tex}  

\begin{document}

%% Seleccionar l'idioma principal del document:
\selectlanguage{spanish}

\beforepreface  

%% RESUM i OVERVIEW
%%%%%%%%%%%%%%%%%%%%%%%%%%%%%%%%%%%%%%%%%%%%%%%%%%%%%%%%%%%%%%%%%%%%%%%%%%%%%
% NOTA: les longituds passades com a parametres d'entrada  s'han
%        d'ajustar manualment fins que el requadre del resum/overview
%        ocupi tota la pàgina. 

%%% Resum en català (o castellà)
\selectlanguage{spanish}
\begin{resum}{10cm}
En los últimos años, las investigaciones sobre los temas de IoT han avanzado notablemente tanto en dispositivos como en tecnologías. La demanda de este tipo de productos hace que muchas empresas cambien su orientación de mercado y se centren en el mundo de los sensores, incluso invierten en hacer una instalación de este tipo para mejorar su productividad.  
\newline 

Las investigaciones no solo se centran en crear nuevos protocolos o nuevas tecnologías, sino también en reutilizar las que ya se conocen, aplicando pequeñas mejoras o sin cambios, ya que son completamente compatibles. 
\newline

Por las razones expuestas surge este proyecto, en el que se realizará un nuevo producto que permitirá a la empresa AlterAid disponer de toda sus infraestructura de manera remota. De este modo podrá desplegar sus aplicaciones en un lugar en el que el acceso o la conexión a internet sean nulos y/o hayan ocurrido desastres naturales, etc. Paralelamente, la investigación sobre nuevas tecnologías para desplegar aplicaciones, utilizadas sobretodo en el ámbito de servidores, serán llevadas al mundo de IoT para solucionar problemas como el alto consumo recursos o los ocasionados al hacer una actualización del sistema o al hacer nuevos despliegues y, a su vez, reducir los tiempos de éstos.
\newline

En el proyecto se utilizará una Raspberry Pi para simular un nodo central o sumidero de datos en la que, mediante Docker, se desplegará toda la infraestructura del core de la empresa AlterAid, Aaaida. Se utilizarán sensores que, mediante bluetooth, se conectarán a la Raspberry Pi para establecer la red, que permitirá el envío de los datos y su correcto almacenamiento para ser gestionados por la aplicación.

\end{resum}

%%% Resum en anglès
\selectlanguage{english}   
\begin{overview}{11cm}
In recent years, the progress on IoT-related issues has seen a very high increase in the effort on research, both for devices and technologies. The high demand for these products is forcing many companies to change their market orientation and focus on the world of sensors, or even to invest in facilities related to this topic in order to improve productivity.
\newline 

Research is not only focused on creating new protocols or new technologies, but also on reusing some which are already known, by applying slight improvements or just keeping them unchanged, as they are fully compatible.
\newline 

For the reasons presented above, this project is born, in which a new product for a company will be developed, in such a way that the company will have all its infrastructure accessible remotely. Therefore, it will be possible to deploy their applications in a place where, for example, Internet access or connection is null, natural disasters have occurred, etc. In parallel, the research of a new technology to deploy applications will be studied too, being it specifically used in the field of servers, but also taken to the world of IoT to solve problems such as the high resources consumption when doing a system update, making new deployments and reducing their times.
\newline 

In the project, a Raspberry Pi is used to simulate a central node or sink of data which, by using Docker, can deploy all the core infrastructure of the AlterAid enterprise (Aaaida). Some sensors will also be used, and by means of Bluetooth they connect to the Raspberry Pi, establishing a network which will allow the sending of data and its proper storage to be managed by the corresponding applications.

\end{overview}

% Tornar a l'idioma principal del document
\selectlanguage{spanish}  

%NOTA: En cas d'utilitzar l'espanyol com a idioma principal del document, el
%      latex anomena les taules com a 'Cuadros'. Si es desitja canviar aquesta
%      nomenclatura i utilitzar la paraula 'Tabla' descomentar les línies següents:
%\def\listtablename{Índice de tablas}
%\def\tablename{Tabla}%



% Amb aqueta comanda indiquem que ja s'han inclòs tots els apartats del prefaci del 
% projecte o podem començar a incloure els capitols de la memòria
\afterpreface


%%%%%%%%%%%%%%%%%%%%%%%%%%%%%%%%%%%%%%%%%%%%%%%%%%%%%%%%%%%%%%%%%%%%%%%%%%
%%%%%% INCLOURE A PARTIR D'AQUÍ TOTS ELS CAPÍTOLS DE LA MEMORIA   %%%%%%%%
%%%%%%%%%%%%%%%%%%%%%%%%%%%%%%%%%%%%%%%%%%%%%%%%%%%%%%%%%%%%%%%%%%%%%%%%%%

% NOTA: recordar que la introducció i les conclusions són capítols NO
%       enumerats, per tant s'ha d'usar \chapter*

% NOTA: és aconsellable incloure els capítols de la memòria en fitxers 
%       separats utlitzant la comanda \input  Per exemple:
%       \input{capitol1}  
%       que farà que s'inclogui el fitxer capitol1.tex

% NOTA: Si es vol incloure agraïments i/o glosari, fer-ho utilitzant 
% \chapter*{} i incloure'ls abans la introducció

\cleardoublepage
\phantomsection
\chapter*{Introducció}
En los últimos años, el avance sobre los temas de IoT han llevado a crecimiento muy elevado en la investigación tanto como dispositivos como de tecnologías para el cumplimiento de todas las barreras que nos ponían el software y el hardware que disponíamos hasta conseguir protocolos muy simples y sensores extremadamente pequeños y baratos. La demanda de este tipo de productos hace que muchas empresas cambie su orientación de mercado y se centren en el mundo de los sensores o incluso invierten en hacer una instalación de este tipo para mejorar su productividad. 

Por eso se decidió realizar este proyecto, el cual se llevará a cabo el despliegue de la plataforma aaaida en una Raspberry. La cual nos puede permitir el uso de dicha plataforma y sus aplicaciones en lugares a los cuales la conexión a internet es nula o escasa, hayan sufrido un desastre natural... ya que es una plataforma centrada en Ihealth que nos permite la monitorización del estado de un paciente.
Gracias a las conexiones de la Raspberry podremos realizar una red con sensores que mediante Bluetooth se podrán comunicar y almacenar los datos. 

Por otra parte el despliegue de la aplicación se llevará a cabo usando Docker, un software que permite contenerizar todo aquello que nuestra aplicaciones necesiten. El cual no sera muy util ya que nos permite instalar o actualizar contenedores por separado y no todo el sistema cada vez. Docker es utilizado básicamente en el mundo de servidores y no de IoT lo cual las ventajas que nos proporciona a la hora del despliegue son contrarrestadas con su complejidad en sistemas con capacidades limitadas.

Nuestra intención es poder desplegar toda la infraestructura de la empresa utilizando un sistema nuevo para IoT, comprobar su viabilidad y beneficios, ofreciendo un producto el cual podría ser de gran utilidad en caso de desastres como seria un sistemas de monitorización de paciente. 

La utilización de la Raspberry hace que esta infraestructura no sea extrapolable al cien por cien de los despliegues de IoT ya que sus especificaciones son bastantes superiores a las de un sensores utilizados normalmente pero como aproximación y prueba piloto para  las tecnologías nombradas anteriormente podrían abrirse un hueco en el mundo es suficiente. 

\chapter{Visión General del proyecto}

El propósito de este proyecto es desplegar toda la infraestructura de la empresa AlterAid en un dispositivo móvil, en nuestro caso una Raspberry Pi, con la finalidad de poder conseguir desplegar las aplicaciones en lugares remotos, sin conexión a internet y/o que han sufrido un desastre natural.

Como segundo objetivo del proyecto queremos implementar un pequeño despliegue de sensores haciendo que nuestra Raspberry Pi sea el nodo central o sumidero de datos. Para poder llevar a cabo todo esto es necesario contar con la utilización de los contenedores usando la tecnología Docker. Puesto que éstos nos facilitarán el trabajo, su utilización fuera del mundo de servidores es motivo de investigación.

\section{Docker}

La idea detrás de Docker es crear contenedores ligeros y portables para las aplicaciones software que puedan ejecutarse en cualquier máquina con Docker instalado, independientemente del sistema operativo que la máquina tenga por debajo, facilitando así también los despliegues.

\begin{figure}[htb]
\begin{center}
\includegraphics[width=0.5\textwidth]{./setup/dockerLogo}
\caption{Logotipo de Docker}
\label{F:prova}
\end{center}
\end{figure}


\section{Aaaida}
Es una plataforma desarrollada por la empresa AlterAid que nos permite crear un usuario y con este usuario poder tener varios vínculos. Un Vínculo es un ente que representa aquello por el que el Usuario se ocupa y preocupa. Ejemplos de Vínculos de Usuario pueden ser familiares, pacientes, amigos, etc. 
Ésta es una plataforma web que recoge todos los datos de las aplicaciones de la empresa y los almacena. Por eso la importancia de poder desplegar toda la infraestructura en un dispositivo que sea fácil de transportar y poder llegar a cualquier lugar.
\newpage
\begin{figure}[htb]
\begin{center}
\includegraphics[width=0.5\textwidth]{./setup/aaaidaLogo}
\caption{Logotipo de Aaaida}
\label{F:prova}
\end{center}
\end{figure}





\section{Motivación}

Conseguir el despliegue de toda la infraestructura de una empresa en un dispositivo de reducidas prestaciones y tamaño como seria una Raspberry Pi permitiría la apertura de nuevos campos de investigación y de mercado puesto que se podrían desarrollar otro tipo de aplicaciones para casos de emergencia o lugares con pocos medios. 
Utilizar una tecnología de servidores como es Docker conlleva un gran avance en el mundo de Internet of Things (de ahora en adelante IoT).
El presente proyecto representa una prueba de concepto para versiones futuras en las cuales se podrían utilizar estas tecnologías para facilitar el despliegue tanto de las aplicaciones como de sus actualizaciones ya que al utilizar Docker solo se debería actualizar el contenedor independiente y no todo el sistema.
\section{Objetivos}

Los objetivos fijados para el desarrollo de este proyecto son los siguientes:

\begin{itemize}
\item Analizar y escoger las tecnologías a utilizar para el desarrollo del proyecto
\item La instalación de Docker en una Raspberry Pi
\item El despliegue de la infraestructura de la empresa AlterAid
\item La creación de pequeños nodos
\item El despliegue de la red
\item Contemplar la viabilidad de creación de redes smart con estas tecnologías. 
\end{itemize}
\pagebreak

\section{Organización del proyecto}

En primer lugar, para poder cumplir todos los puntos de los objetivos necesitaremos estudiar un poco más a fondo todas las tecnologías que se utilizarán a lo largo de todo el proyecto. Serán explicadas en sus respectivos capítulos.

Como se ha explicado previamente se utilizará una Raspberry Pi, donde se desplegará la aplicación mediante Docker. Para poder llevarlo a cabo se tendrán que estudiar las diferentes posibilidades como, por ejemplo, si es posible ejecutar Docker en un sistema ARM o si se podrá virtualizar la Raspberry para poder hacer las pruebas de una manera más cómoda. Todo esto se podrá ver en el capítulo 2 y capítulo 3.

Una vez desplegada y ejecutada la aplicación es necesario arrancar Aaaida. En el capítulo 4 se podrá ver una pequeña explicación del funcionamiento de la plataforma y sus funcionalidades. 

Para terminar la parte técnica, vendría el paso de crear la red de sensores y
comunicarnos con la Raspberry. Una vez se haya llevado a cabo esta conexión, se realizará el envío de los datos y su correcto almacenamiento. En el capítulo 5 se comentará como hacerlo para poder ser visualizados en la plataforma Aaaida. 

Por último, con la plataforma en la Raspberry y la red de sensores funcionando obtendremos el último capítulo donde se podrán ver las conclusiones y resultados sobre la viabilidad de este proyecto.

\chapter{Virtualización}

\section{Qué es la virtualización}

La virtualización es la creación de una versión virtual basada en software de algo, en lugar de una física. Se puede aplicar a sistemas operativos, almacenamiento, servidores, aplicaciones, redes… y es una manera de reducir gastos, aumentar eficiencia y agilidad en las empresas.  

\subsection{Tipos de virtualización}

Estos son los 4 tipos de virtualización más habituales.
\subsubsection{Virtualización de servidores}

La virtualización en servidores ayuda a evitar ineficiencias ya que permite ejecutar varios sistemas operativos en una máquina física como máquinas virtuales, con acceso a los recursos de todos. 
También permite generar un cluster de servidores en un único recurso para así mejorar mucho más la eficiencia y la reducción de costes. También permite el aumento de rendimiento de las aplicaciones, la disponibilidad al aumentar la velocidad en la carga de trabajo.

\subsubsection{Virtualización de escritorios}

La implementación de escritorios virtualizados permite ofrecer a las sucursales o empleados externos… de forma rápida y sencilla un entorno de trabajo y una reducción de la inversión a la hora de gestionar cambios en estos. 

\subsubsection{Virtualización de red}

Se trata de reproducir una red completa física mediante software, para poder ejecutar los mismo servicios que una red convencional y dispositivos. Cuentan con las misma características y garantías que las redes físicas con las ventajas que nos ofrece la virtualización más la liberación del hardware.

\subsubsection{Almacenamiento definido por software}

La virtualización del almacenamiento permite prescindir de  los discos de los servidores, los combina en depósitos de almacenamiento de alto rendimiento y los distribuye como software. Este nuevo modelo es permite aumentar la eficiencia en el guardado de datos.
\pagebreak  

\subsection{Ventajas de la virtualización}

Como se ha podido apreciar en los tipos de virtualización, esta conlleva una mejora considerable tanto en el rendimiento, agilidad, flexibilidad, escalabilidad… Como en una reducción de costes considerables tanto en tiempo como monetarios y una simplificación en la gestión de la infraestructura.

\begin{itemize}
\item Reduce los costes de capital y los gastos operativos.
\item Minimiza o elimina los tiempos de inactividad.
\item Aumenta la productividad, la eficiencia, la agilidad y la capacidad de respuesta.
\item Implementa aplicaciones y recursos con más rapidez.
\item Garantiza la continuidad del negocio y la recuperación ante desastres.
\item Simplifica la gestión del centro de datos. 
\end{itemize}

\section{Qué es Docker}

La idea detrás de Docker es crear contenedores ligeros y portables para las aplicaciones software que puedan ejecutarse en cualquier máquina con Docker instalado,
independientemente del sistema operativo que la máquina tenga por debajo, facilitando así también los despliegues.

De una manera más sencilla Docker los que nos proporciona es la opción de poder meter en pequeños contenedores todo aquello que nuestra aplicación necesite y poder desplegarla en cualquier maquina que tenga instalado Docker sin preocuparnos de nada más. 

Se podría decir que son pequeñas “máquinas virtuales” pero muchos más ligeras ya que utilizan el sistema operativo de donde se ejecuta y el contenido relevante para ejecutar la aplicación está dentro de los contenedores.
\newline 

Docker es:

\begin{itemize}
\item Open-Source para la gestión de "virtualización de contenedores"
\item Aísla múltiples sistemas de archivos en el interior del mismo host
\begin{itemize}
\item Las instancias se llaman Contenedores
\item Te dan la ilusión de estar dentro de una máquina virtual
\end{itemize}
\item Piensa en entornos de ejecución o "sandboxes"
\item No hay necesidad de un hypervisor (rápido de ejecutar)
\item Requiere x64 y Linux kernel 3.8+
\end{itemize}
\pagebreak 
Docker no es:

\begin{itemize}
\item Un lenguaje de programación
\item Un sistema operativo
\item Una máquina virtual
\item Una imagen en el concepto tradicional de la máquina virtual basada en hipervisor
\end{itemize}

\section{Máquinas Virtuales vs Docker}

Las máquinas virtuales, incluyen toda la aplicación, los binarios y librerías necesarias, y todo un sistema operativo cosa que hace que que ocupen mucho espacio, el tiempo de ejecución sea lento, la necesidad de un Hipervisor para su utilización.

Por lo contrario Docker container incluyen la aplicación y todas sus dependencias pero comparten el núcleo con otros contenedores, funcionando como procesos aislados en el sistema de ficheros del sistema operativo. Docker container no están vinculados a ninguna infraestructura específica: se ejecutan en cualquier ordenador, en cualquier infraestructura, y en cualquier cloud.

\begin{figure}[htb]
\begin{center}
\includegraphics[width=1\textwidth]{./setup/VrvsDocker}
\caption{Comparativa de máquina virtual y Docker}
\label{F:VrvsDocker}
\end{center}
\end{figure}

En la imagen se puede apreciar la diferencias comentadas anteriormente, mientras la primera columna sería la virtualización de 3 aplicaciones se puede ver que hay la capa intermedia del Hipervisor el cual nos permite ejecutar los sistemas operativos virtualizados, todos los binarios y librerías que requieren cada aplicación y finalmente la aplicación. 
En la segunda columna tenemos la contenerización de 3 aplicaciones, podemos observar que no hace falta un hipervisor ya que utilizan el sistema operativo de la infraestructura donde se ejecuta, pero si son necesarios los binarios y las librerías.
A simple vista se puede apreciar que si eliminamos el sistema operativo la infraestructura completa se hace más liviana y rápida de ejecutar.

\section{Porque Docker?}

Por lo motivos listados en el apartados anteriores se decidió en utilizar Docker para realizar el despliegue de las aplicaciones, ya que se necesita de una tecnología la cual pueda ser almacenada en dispositivos con una memoria reducida en este caso docker la cumple. 

Su rapidez a la hora de levantar el servicio, en el mundo de IoT el tiempo es un bien preciado y los sistemas se pueden apagar y encender constantemente para evitar gastos de energía innecesarios.

La facilidad para poder desplegar los servicios, con Docker desplegar servicios es muy sencillo solo requiere tenerlo instalado y ejecutar el contenedor para que el servicio esté activo.

La sencillez a la hora de mantener el sistema, si hay que hacer actualizaciones o controles de una pequeña parte del servicio solo habria que cambiar o actualizar ese contenedor no todo el sistema. Esto es un gran ahorro en recursos y tiempo. 

Por todo esto se piensa que Docker puede ser una buena tecnología para desplegar sistemas de Iot. También se tiene en cuenta que la utilización de una Raspberry 3 no es un aparato con unas capacidades limitadas como podría ser un sensor utilizado normalmente pero si que puede validar todas las funciones listadas anteriormente y servir como preámbulo para la utilización en el resto de despliegues.  


\chapter{Raspberry Pi}

\section{¿Por qué Raspberry Pi?}

La utilización de una Raspberry Pi en este proyecto son claras y comentadas con anterioridad. 
Es un dispositivo de un tamaño muy reducido y portátil, que nos facilita mucho el poder llevarla a cualquier lugar sin dificultades y poder ofrecer un servicio en cualquier lugar.
Tiene unas prestaciones más que aceptables para un dispositivo de ese tamaño, conexiones inalámbricas como serian Wifi y Bluetooth integrados cosa impredecible para las comunicaciones con los sensores. Todo a un precio muy atractivo el cual permite su compra.  

Algunas de las especificaciones mas importantes son las siguientes:
\begin{itemize}
\item 1.2GHz 64-bit quad-core ARMv8 CPU
\item 802.11n Wireless LAN
\item Bluetooth 4.1 y Bluetooth Low Energy (BLE)
\item 1GB RAM
\item Ethernet port
\item Micro SD card slot 
\end{itemize}

\section{Virtualización de la Raspberry}

Al inicio del proyecto no se disponía de una Raspberry para poder realizar las pruebas y se decidió emular la mediante qemu. 
Qemu es una aplicación que nos permite emular mediante máquinas virtuales gran parte de los sistemas operativos. 
Lo cual es realmente sencillo emular una imagen de Raspbian si no fuera por un detalle, como se comenta en el apartado ¿Qué es Docker? Docker necesita un sistema x64 o Linux kernel 3.8+ y Raspberry ejecuta un sistema ARM lo que conlleva que Docker no sea compatible a primera instancia con la Raspberry. 
La solución para este gran problema es la utilización de Hypriot, una imagen de Raspbian modificada con Docker instalado. 
La instalación de esta imagen  requiere hacer unos retoques ya que no es del todo compatible el kernel de qemu con el de la imagen de Hypriot. Cosa que aun que permite emular perfectamente la imagen con Docker, no nos permite el uso correcto. Por lo tanto se decidió apartar por completo la posibilidad de poder emular una Raspberry con Docker instalado ya que las incompatibilidades del kernel no lo permitían.

En el apéndice se podrá ver los pasos a seguir para la instalación y emulación de la imagen. 

\section{Raspberry y Docker}

Después de descartar completamente la opción de emular la imagen de la Raspberry se decidió probar en la Raspberry que disponía la empresa la imagen corria perfectamente. Y si, la imagen iba perfectamente y ejecutaba Docker sin ningún problema. Ahora el problema era la Raspberry, era el primer modelo el cual era viejo y no disponía de módulos para conexiones inalámbricas integrados por lo que se decidió aprovechando el lanzamiento de la nueva Raspberry 3 comprarla.

\chapter{Aaaida}

En este capítulo se presentará el servicio de Aaaida utilizado para la creación de este proyecto. 

\section{Arquitectura}

La arquitectura de Aaaida sería la que podemos ver en la figura \ref{a:arquitectura} donde también podemos ver las tecnologías que utilizan cada una de las partes. Todas las partes que la componen utilizan JavaScript conocido como stack MEAN (MongoDB-Express-AngularJS-Node.js). 

\begin{figure}[htb]
\begin{center}
\includegraphics[width=1\textwidth]{./setup/arquitectura}
\caption{Arquitectura de Aaaida}
\label{a:arquitectura}
\end{center}
\end{figure}

\subsection{Stack MEAN}

El llamado stack MEAN es el desarrollo end-to-end basado en Javascript en
cada una de sus partes, de esta manera, se permite desarrollar todo lo necesario
sobre la infraestructura de JavaScript tal y como se puede ver en la figura \ref{mean:MEAN}

\begin{figure}[htb]
\begin{center}
\includegraphics[width=0.25\textwidth]{./setup/MEAN}
\caption{Arquitectura de Aaaida}
\label{mean:MEAN}
\end{center}
\end{figure}

\pagebreak
El hecho de que se utilice un mismo lenguaje de programación (JavaScript) para
cada una de las tecnologías, permite que una persona pueda manejarse en
cualquier ámbito, manteniendo una colaboración continua en los proyectos a
desarrollar.

\subsection{API REST}

Una API Rest es una librería de funciones a la que se accede mediante el
protocolo http, es decir, a través de direcciones web o URLs en las que se envían
las consultas necesarias para acceder a la información que hay en la base de
datos. Como respuesta a la consulta, se obtienen diferentes formatos que
pueden ser textos planos, objetos JSON, entre otros.

Esta API está formada por los siguientes componentes y tecnologías:

\subsubsection{MongoDB}

Mongo es una base de datos no relacional (NoSQL) de código abierto que
guarda cada uno de los datos en documentos JSON ( JavaScript Object
Notation) de forma binaria para que la integración sea más rápida. 
Orientado a documentos, de esquema libre, esto significa que cada entrada o registro puede tener un esquema de datos diferente, con atributos o “columnas” que no tienen por qué repetirse de un registro a otro.

\subsubsection{Node.js}

Node.js es un framework en JavaScript que utiliza el motor de Google
denominado V8 y proporciona las funcionalidades core de una aplicación,
mediante una arquitectura orientada a eventos asíncronos (APIs no-
bloqueantes) que le permiten un gran rendimiento y escalabilidad.
Aunque se puede utilizar para crear cualquier tipo de lógica de aplicación, el
hecho de que disponga de un módulo para poder actuar como servidor web,
hace que sea unos de lo más utilizados en el desarrollo de aplicaciones web.

\subsubsection{Express}

Express es un framework en JavaScript para Node.js que permite crear
servidores web y recibir peticiones http, de manera sencilla y eficiente.
El objetivo principal de Express, es el de ofrecer soporte en diferentes
necesidades, tales como: gestión de peticiones y respuestas, cabeceras...

\subsection{Consola de Aaaida}

La consola de Aaaida es una plataforma web desarrollada por Alteraid que permite administrar y controlar toda aquella información relacionada con el ecosistema de Aaaida.

Este panel de administración ha sido implementado utilizando las siguientes
tecnologías:

\subsubsection{AngularJS}

AngularJs es un framework mantenido por Google con JavaScript para la parte de cliente o frontend en una aplicación web. Este utiliza el patrón de diseño MVC (Modelo-Vista-Controlador) permitiendo crear SPAs (Single-Page Applications).

El hecho de que una aplicación web quepa en una sola página, proporcionando una experiencia fluida a cada uno de los usuarios. El patrón de arquitectura separe datos y lógica, permite el desarrollo este tipo de aplicaciones de una manera más flexible, convirtiéndose en una de las tecnologías más
utilizadas.

\subsubsection{Web 3.0}

Web 3.0 es una web con la que interactuar para conseguir resultados, permitiendo compartir información por cada persona de forma inteligible y diseñada de manera eficiente con tiempos de respuesta optimizados. Este tipo de web facilita la accesibilidad de las personas a la información sin depender de qué tipo de dispositivo se use.
\chapter{Implementación}

En este capítulo se explicará las implementaciones realizadas para poder llevar a cabo el proyecto, tanto como para conseguir la comunicación con el sensor utilizado, los cambios realizados en Aaaida para poder visualizar las mediciones del sensor y por último el despliegue en la Raspberry Pi. 

\section{Comunicación con el sensor}

Para la realización del proyecto es necesario un sensor el cual se pueda comunicar y enviar los datos por bluetooth. En la empresa se dispone de una serie de sensores médicos de los cuales se utilizará un monitor de ritmo cardiaco, Zephyr BioHarness 3. 

\begin{figure}[htb]
\begin{center}
\includegraphics[width=0.5\textwidth]{./setup/zephyr}
\caption{Sesor Zephyr BioHarness 3}
%\label{a:arquitectura}
\end{center}
\end{figure}

El fabricante te ofrece gran cantidad de documentación y una aplicación de prueba para los desarrolladores que quieran realizar productos y utilicen sus sensores. Pero todo está orientado a aplicaciones Android, las cuales son las más populares para este tipo de sensores. 

Por lo cual después de buscar y ponerse en contacto con la empresa no hay ningún tipo protocolo para establecer contacto y recibir los datos en JavaScript. Por lo tanto, se decidió realizar uno utilizando toda la documentación y ejemplos para otros lenguajes. 

Para la realización del código de comunicación fueron necesarios 2 módulos de Node.js uno que nos calcula el CRC y otro que establece una conexión bluetooth.

Módulos utilizados:

\begin{itemize}
\item crc
\item bluetooth-serial-port 
\end{itemize}
\pagebreak

\subsection{Protocolo}

El código realizado para poder establecer la comunicaciones fue el siguiente.
 
Se cargan los módulos externos y se declaran las variables, la dirección MAC del sensor se pone a clavo para evitar interferencias con otros dispositivos bluetooth.

\begin{verbatim}
var crc = require('crc');
var btSerial = new (require('bluetooth-serial-port')).BluetoothSerialPort();

ADDRESS = "E0:D7:BA:A7:F1:5D";
var results = [];
var is_stopping = false;
\end{verbatim}

Función \texttt{connect}, como su nombre indica, nos conectara con el sensor y empezará a recibir datos. 

\begin{verbatim}
function connect(callback) {
   var socket = btSerial.on('found', function (address) {
       if (address == ADDRESS) {
           btSerial.findSerialPortChannel(address, function (channel) {
               btSerial.connect(address, channel, function () {
                   console.log('connected to ' + address);
                   btSerial.on('data', function (buffer) {
                       decode(buffer);
                   });
                   listener(socket, function (res) {
                       callback(res);
                   });
               }, function () {
                   console.log('cannot connect');
               });
           }, function () {
               console.log('found nothing');
           });
       }
   });
   btSerial.inquire();
}
\end{verbatim}

La función \texttt{decode}, nos permite decodificar de una manera muy simplificada los bytes que recibimos. Le pasamos el segundo byte y según su valor podremos saber qué tipo de mensaje nos está enviado el sensor. Si en el segundo byte que se recibe es un 44 implica que en el octavo byte se está recibiendo el ritmo cardiaco.

\begin{verbatim}
function decode(data) {
   switch (data[1]) {
       case 35:
           console.log("Received LifeSign message");
           break;
           \end{verbatim}
           %\pagebreak
           \begin{verbatim}
       case 44:
           console.log("Received Event message");
           results.push(data[8]);
           break;
       case 43:
           console.log("Received Summary Data Packet");
           break;
       case 37:
           console.log("Received Accelerometer Data Packet");
           break;
       case 36:
           console.log("Received R to R Data Packet");
           break;
       case 33:
           console.log("Received Breathing Data Packet");
           break;
       default:
           console.log("Packet type: " + data[1]);
           console.log("Received Not recognised message");
           break;
   }
}
\end{verbatim}

Una vez conectados al sensor se debe enviar mensajes a este para que mantenga la conexión y no se desconecte (\texttt{lifeSings}). Como la finalidad es realizar una medida la comunicación se realizará durante 20 seg, una vez pasados estos 20 seg se cerrara la conexion. 

\begin{verbatim}
function listener(socket, callback) {
   socket.on('data', function (buffer) {
       decode(buffer);
       lifeSing = create_message_frame('100011', 0);
       socket.write(new Buffer(lifeSing), function (err, bytesWritten) {
           if (err) console.log(err);
       });
   });
   setTimeout(function () {
       stop(socket, function (res) {
           callback(res);
       });
   }, 20000);
}
\end{verbatim}

La creación de los mensajes \texttt{lifeSings} se realizan de la siguiente manera: Son la concatenación de un byte de sincronismo, el mensaje, el dlc que será la longitud del payload, el crc calculado mediante el payload y por ultimo otro byte de cierre. Todo paquete se pasará a hexadecimal y se procederá a enviarlo.
 \pagebreak
\begin{verbatim}
function create_message_frame(message_id, payload) {
   dlc = payload.toString().length;
   if (0 <= dlc <= 128) {
       crc_byte = crc.crc32(payload);
       message_bytes = '00000010'+ message_id + dlc + payload + crc_byte 
       + '00000011';
       message_fame = Bin2Hex(message_bytes);
       return message_fame
   }
}
\end{verbatim}

Por último la función de \texttt{stop} y la función de \texttt{avg}, la función de \texttt{stop} cerrará la conexión y la función de \texttt{avg} calcula la media de todas las medidas tomadas durante los 20 seg y devolverá la media. 

\begin{verbatim}
function stop(socket, callback) {
   is_stopping = true;
   socket.close();
   avg(function (res) {
       callback(res);
   });

}

function avg(callback) {
   var sum = 0;
   if (is_stopping == true) {
       for (var i = 0; i < results.length; i++) {
           sum = sum + results[i];
       }
       var media = sum / results.length;
       callback(media);
   }
}
\end{verbatim}

Con esto podremos establecer una conexión con el sensor y poder recibir la media del pulso cardiaco durante 20 seg. Hay que tener en cuenta la simplicidad del protocolo ya que solo capturamos una de las funciones (ritmo cardíaco) que nos proporciona el sensor Zephyr ya que si quisiéramos poder obtener todos los datos la complejidad sería mucho mayor y para desarrollarlo en JavaScript se necesitaría mucha más información sobre cómo se envían las tramas y que contiene cada una de ellas. 
\pagebreak
\section{Integración con Aaaida}

Como se explicó en el capítulo anterior la consola de Aaaida es una plataforma web, que nos permite administrar toda la información de las aplicaciones vinculadas a ella. 
Para poder incluir nuestra aplicación de monitorización del ritmo cardiaco con Aaaida y así visualizar lo en la web hay que crear un plugin de Aaaida con nuestro proyecto. 

\subsection{Creación de un plugin}

Para el desarrollo de cada uno de los plugins se ha utilizado el framework llamado route-injector desarrollado por Alteraid y Ondho. 

El objetivo principal de este framework es el de generar, de manera automática:
los método http CRUD de la API REST (GET, PUT, POST,DELETE) y una
backoffice, mediante modelos.

Para la creación del plugin de proyecto deberemos de seguir la arquitectura principal de los plugins.


\begin{figure}[htb]
\begin{center}
\includegraphics[width=0.20\textwidth]{./setup/arc}
\caption{Arquitectura de los plugins}
%\label{a:arquitectura}
\end{center}
\end{figure}

\subsubsection{Models}

En el caso de este proyecto no se creará ningún modelo de datos nuevo, ya que Aaaida dispone de un conjunto de modelos de datos que concuerda con el modelo que necesitamos. El modelo elegido será Value, que cumple todas las necesidades de las medidas realizadas.

Los campos utilizados son los siguientes:

\begin{verbatim}
{
    value: {type: String},
    user: {type: mongoose.Schema.Types.Mixed, ref: 'User'},
    bond: {type: mongoose.Schema.Types.Mixed, ref: 'Bond'},
    measure: {type: mongoose.Schema.Types.Mixed, ref: 'Measure'},
    tags: {type: String},
    measured_at: {type: Date, readonly: true}
}
\end{verbatim}
\pagebreak

\begin{itemize}
\item value: Será el valor de la medida de ritmo cardiaco registrada. 
\item user: El usuario con el cual se está logueado en Aaaida
\item bond: Sería el “paciente” al cual se le toma la medida.
\item tags: Este campo se utiliza para diferenciar de a qué aplicación pertenecen los valores.  
\item measured at: El instante que se tomó el valor. 
\end{itemize}

\subsubsection{Routes}

En el caso de que fuese necesario obtener información más concreta sobre el
modelo de datos, se deberían de definir cada una de esas rutas en la carpeta
routes del plugin. Como hemos dicho route-injector nos genera automáticamente el CRUD para el modelo Values, pero necesita una ruta específica para ejecutar la conexión vía bluetooth con el sensor. 

Por lo tanto fue necesario la creación de la ruta, en el mismo fichero se copio todo el protocolo para la conexión con el sensor Zephyr. 

\begin{verbatim}
module.exports.route = function (router) {
   router.get('/coiote/media', function (req, res) {
       connect(function (media) {
           console.log("Tu HR media = " + media);
           res.json(media)
       });
   });
};
\end{verbatim}

Como se puede apreciar en el código una petición \texttt{get} a \texttt{ /coiote/media } ejecutará la función \texttt{connect} que establece la comunicación y recolección de datos.  

\subsubsection{Index}

En el siguiente fichero se configuran las funciones iniciales del plugin una vez
este ha cargado.

\begin{verbatim}
module.exports.config = require('./plugin.json');

module.exports.init = function (conf) { 
};
\end{verbatim}

No plugin no debe hacer ninguna función al iniciarse por lo tanto no debemos añadir nada en este fichero.
\pagebreak
\subsubsection{Package} 

En este fichero, se describe toda la información necesaria del paquete. 

\begin{verbatim}
{
 "name": "coiote-plugin",
 "version": "0.0.1",
 "description": "Plugin for CoIoTe",
 "dependencies": {
   "bluetooth-serial-port": "^2.0.0",
   "crc": "^3.4.0"
 }
}
\end{verbatim}

Se puede apreciar las 2 dependencias a módulos externos nombrados anteriormente. 

\subsubsection{Plugin} 

En este fichero se indica el nombre del plugin y el directorio donde se encuentran
las rutas estáticas. Sirve para especificar dónde se encuentran todas esas rutas generadas para el plugin y no han sido creadas automáticamente por route-injector. 

\begin{verbatim}
{
 "name": "CoIoTe",
 "routes": ["routes"]
}
\end{verbatim}

\subsection{Creación de la página}

Una vez el plugin está creado, se necesitará crear la página web donde poder visualizar en la consola de Aaaida los resultados obtenidos de las medidas. 

Para esto se seguirá un proceso similar al de crear un plugin, pero en el directorio de pages, que tiene una arquitectura base similar a la siguiente:  

\begin{verbatim}
FOTO DE LOS DIRECTORIOS!!!!!!!!!
\end{verbatim}

En este directorio, se encuentran definidos cada uno de los templates externos que se añadirán a la backoffice.

En este caso, ha sido necesario añadir el de coiote. En este directorio podemos encotrar el template index.js donde se define el controlador y como se visualiza en la consola de Aaaida y su Url. 

\begin{verbatim}
module.exports = {
   backoffice: true,
   url: 'coiote/mesures',
   template: 'coiote/index.html',
   controller: 'ChartCoioteController',
   menu: {
       clickTo: 'coiote/mesures',
       title: "mesures",
       section: "CoIoTe"
   }
   //backoffice: false // standalone website
};
\end{verbatim}

Creandonos una pestaña en la consola de Aaaida para el proyecto como podemos ver en la figura \ref{sec:coioteSec}. 

\begin{figure}[htb]
\begin{center}
\includegraphics[width=0.5\textwidth]{./setup/arquitecturaCoiote}
\caption{Sección en la consola de Aaaida}
\label{sec:coioteSec}
\end{center}
\end{figure}

Una vez creado este fichero pasaremos a crear la template index.html y su controlador controller.js. En el controlador se realizan petición a las rutas creadas por route-injector para el modelo value y la ruta creada en el directorios routes para establecer la comunicación, y así poder rellenar el contenido de la plantilla.

La página para la visualización de los resultados seria esta:

\begin{figure}[htb]
\begin{center}
\includegraphics[width=1\textwidth]{./setup/visualizacionPaginaCoiote}
\caption{visualización de los datos en Aaaida}
%\label{sec:coioteSec}
\end{center}
\end{figure}

\cleardoublepage
\phantomsection
\chapter*{Conclusiones}

Escriure aquí les conclusions del projecte. 

\chapter*{Glosario}

{\bf Journalig}: Memoria caché en la cual se tiene constancia de las escrituras hechas en disco. Esto en caso de fallo o desconexión no prevista ayuda a evitar la corrupción de datos. Por esta razón se activa, ya que la Raspberry Pi es un dispositivo sin batería y si hay un corte de luz, puede que se corrompa la base de datos y el restar no se pueda ejecutar. 

{\bf Daemon}: Proceso en segundo plano que se inicia como servicio.

{\bf API}: Application Programming Interface. Es un conjunto de especificaciones
para la comunicación entre diferentes componentes software.

{\bf IoT}: Internet Of Things

{\bf JSON}: Javascript Object Notation

{\bf MVC}: El modelo vista controlador, es un patrón de arquitectura de software que separa los
datos y la lógica de una interfaz de usuario y el módulo encargado de gestionar cada uno de sus
eventos y comunicaciones.

{\bf SPA}: Single-page Application, es una aplicación web que cabe en una sola página con el
objetivo de proporcionar una experiencia fluida a los usuarios.




%%%  BIBLIOGRAFIA
%%%%%%%%%%%%%%%%%%%%%%%%%%%%%%%%%%%%%%%%%%%%%%%%%%%%%%%%%%%%%%%%%%%%%%%%%%

%%% Per la bibliografia hi ha 2 opcions: generarla amb la utilitat BibTeX 
%%%                                      o fer-la ''a ma''
%%% NOTA: podeu trobar facilment informació sobre BibTeX a:
%%%  http://www.ctan.org/tex-archive/biblio/bibtex/contrib/doc/

%%% OPCIO 1: BibTeX (recomanat) -> descomentar les comandes seguents:
%\bibliographystyle{unsrt}   %% Estil de bibliografia EETAC
%\cleardoublepage
%\phantomsection
% Indicar aqui el(s) fitxer(s) que contenen la bibliografia
%\bibliography{fitxer1,...,fitxerN}  
%\pdfbookmark{Bibliografia}{sec:biblio}

%%% OPCIO 2: bibliografia manual
%%%
%%% L'argument d'entrada es el numero de referencies que s'inclouen
\cleardoublepage
\phantomsection
\begin{thebibliography}{2}

%% Llibres:  Autor/s (cognoms i inicials dels noms), títol del llibre (en cursiva), editor, ciutat i any de publicació. Quan es cita el capítol d'un llibre s'ha d'indicar el títol del capítol (entre cometes), el títol del llibre (en cursiva) i els números de pàgines amb la primera i la darrera incloses.

%%  Exemple de capitol en llibre
\bibitem{prova1} 
Turnbull, J. {\it The Docker book}.
(último acceso: 20 de 06 de 2016)

\bibitem{prova3} 
Humble, J. {\it Continuous Delivery}.
(último acceso: 20 de 06 de 2016)

\bibitem{prova4} 
Newman, S. {\it Building Microservices}.
(último acceso: 20 de 06 de 2016)

\bibitem{prova4} 
{\bf Web de Raspberry Pi}. URL: {\it https://www.raspberrypi.org/}
(último acceso: 17 de 10 de 2016)

\bibitem{prova4} 
{\bf Web de Docker}. URL: {\it https://www.docker.com}
(último acceso: 17 de 10 de 2016)

\bibitem{prova4} 
{\bf Blog de Hypriot}.  URL: {\it http://blog.hypriot.com/}
(último acceso: 17 de 10 de 2016)

\bibitem{prova4} 
{\bf Web de Alteraid}.  URL: {\it http://www.alteraid.com/}
(último acceso: 17 de 10 de 2016)

\bibitem{prova4} 
{\bf Web de MongoDB}.  URL: {\it https://www.mongodb.com/}
(último acceso: 17 de 10 de 2016)

\bibitem{prova4} 
{\bf Repositorio de Angular-chart}.  URL: {\it http://jtblin.github.io/angular-chart.js/}
(último acceso: 17 de 10 de 2016)

\bibitem{prova4} 
{\bf Repositorio de CRC}.  URL: {\it https://www.npmjs.com/package/crc}
(último acceso: 17 de 10 de 2016)

\bibitem{prova4} 
{\bf Repositorio de bluetooth-serial-port }.  URL: {\it https://www.npmjs.com/package/bluetooth-serial-port}
(último acceso: 17 de 10 de 2016)

\bibitem{prova4} 
{\bf Web de Zephyr, development tools }.  URL: {\it https://www.zephyranywhere.com/zephyr-labs/development-tools}
(último acceso: 17 de 10 de 2016)

\bibitem{prova4} 
{\bf Web de Qemu }.  URL: {\it http://wiki.qemu.org/Main Page}
(último acceso: 17 de 10 de 2016)

\bibitem{prova4} 
{\bf Web de Vmware }.  URL: {\it http://www.vmware.com}
(último acceso: 17 de 10 de 2016)

\bibitem{prova4} 
{\bf GitHub }.  URL: {\it https://github.com}
(último acceso: 17 de 10 de 2016)

%%  Exemple de d'article en revista


\end{thebibliography}

%%%%%%%%%%%%%%%%%%%%%%%%%%%%%%%%%%%%%%%%%%%%%%%%%%%%%%%%%%%%%%%%%%%%%%%%%%
%%%%%%                           APENDIXS                         %%%%%%%%
%%%%%%%%%%%%%%%%%%%%%%%%%%%%%%%%%%%%%%%%%%%%%%%%%%%%%%%%%%%%%%%%%%%%%%%%%%
\pagestyle{empty}  % no tocar

%% Descomentar una de les dues línies següents, en funció de:
%%  a) els apendixs s'encuadernaran apart (amb portada) 
%%  b) els apendixs s'enquadernen amb el mateix projecte (sense portada). 
%% Recordeu que si tot el document (amb apèndixs) excedeix les 100 pagines 
%% s'ha d'enquadernar a part
%\appendix\ambportada
\appendix\senseportada


%%%%%%%%%%%%%%%%%%%%%%%%%%%%%%%%%%%%%%%%%%%%%%%%%%%%%%%%%%%%%%%%%%%%%%%%%%
%%%%%% INCLOURE A Psudo apt-get install texmakerARTIR D'AQUI TOTS ELS CAPÍTOLS DELS APENDIXS   %%%%%%%%
%%%%%%%%%%%%%%%%%%%%%%%%%%%%%%%%%%%%%%%%%%%%%%%%%%%%%%%%%%%%%%%%%%%%%%%%%%

\chapter{Emular la Raspberry Pi en Qemu}

\begin{verbatim}
----------Compilar qemu con soporte ARM 1176---------------

# apt-get install git zlib1g-dev libsdl1.2-dev
# apt-get install libpixman-1-dev libfdt-dev
$ mkdir raspberrypi
$ cd raspberrypi
$ git clone git://git.qemu-project.org/qemu.git
$ cd qemu
$ ./configure --target-list="arm-softmmu arm-linux-user" --enable-sdl 
--prefix=/usr
$ make
# make install

---------- Descargar la imagen de la raspberry pi y el kernel-qemu------

$ wget "http://downloads.raspberrypi.org/raspbian_latest"
$ unzip 2016-03-18-raspbian-jessie.img.zip    //(la versión que te descarges)
$ rm -rf 2016-03-18-raspbian-jessie.img.zip	  //(la versión descargada)

$ fdisk -l 2016-03-18-raspbian-jessie.img 
$ sudo mount -v -o offset=67108864 -t ext4 2016-03-18-raspbian-jessie.img /mnt/
$ cd /mnt
$ sudo nano ./etc/ld.so.preload  //(comentar el codigo)
$ sudo nano ./etc/fstab 	 	 //(Comentar la segunda linea)
$ cd ~
$ sudo umount /mnt

------------Ejecutar por primera vez-------------------------

$ qemu-system-arm -kernel kernel-qemu -cpu arm1176 -m 256 -M versatilepb 
-no-reboot -serial stdio -append "root=/dev/sda2 panic=1 rootfstype=ext4 rw" 
-hda 2016-03-18-raspbian-jessie.img -redir tcp:2222::22


------------ Resize de la imagen ---------------------

$ qemu-img resize raspi.img +16024M 
$ qemu-system-arm -M versatilepb -cpu arm1176 -hda 2016-03-18-raspbian-jessie.img 
-kernel kernel-qemu -m 192 -append "root=/dev/sda2" -drive file=raspi.img
$ sudo apt-get -y install gparted
$ sudo gparted
$ qemu-system-arm -kernel kernel-qemu -cpu arm1176 -m 256 -M versatilepb 
-no-reboot -serial stdio -append "root=/dev/sda2 panic=1 rootfstype=ext4 rw"
-hda raspi.img -redir tcp:2222::22
\end{verbatim}

Puede haber problemas con la interfaz gráfica, ya que si no hay suficiente memoria en la imagen, se deshabilita. 

\chapter{Ficheros utilizados}

Todos los ficheros utilizados y modificados para realizar el despliegue. 

\section{Fichero entrypoint}
\begin{lstlisting}[style=Bash]
function wait {
    echo -n "waiting for TCP connection to $1:$2..."
    while ! nc -w 1 $1 $2 2>/dev/null
    do
        echo -n .
        sleep 1
    done
    echo 'ok'
}
function startup {
    list=$(env | grep DOCKER_ | grep _TCP= | cut -d = -f 2)
    elems=($list)
    for key in "${!elems[@]}"
    do
        str=${elems[$key]}
        str2=${str#"tcp://"}
        IFS=:
            array=($str2)
        unset IFS
        host=${array[0]}
        port=${array[1]}
        wait $host $port
    done
}
startup
node bin/www
\end{lstlisting}

\section{Script build rpi image}
\begin{lstlisting}[style=Bash]
#!/bin/bash
getVersion(){
    echo $(node -p -e "require('./package.json').version");
}
echo "Building image for version $(getVersion)"
echo "Installing dependencies"
npm install
echo "Downloading bower components for stats page"
cd stats
bower install --allow-root
cd ..
VERSION=getVersion
docker build -t alteraid/aaaida-datastore-arm -f scripts/rpi_docker/Dockerfile .
\end{lstlisting}

\section{Dockerfile modificado}
\begin{lstlisting}[style=Bash]
FROM ioft/armhf-debian
RUN apt-get update; apt-get -y install curl; 
apt-get install libbluetooth3; apt-get -y install bluez

RUN set -ex \
  && for key in \
    9554F04D7259F04124DE6B476D5A82AC7E37093B \
    94AE36675C464D64BAFA68DD7434390BDBE9B9C5 \
    0034A06D9D9B0064CE8ADF6BF1747F4AD2306D93 \
    FD3A5288F042B6850C66B31F09FE44734EB7990E \
    71DCFD284A79C3B38668286BC97EC7A07EDE3FC1 \
    DD8F2338BAE7501E3DD5AC78C273792F7D83545D \
    B9AE9905FFD7803F25714661B63B535A4C206CA9 \
    C4F0DFFF4E8C1A8236409D08E73BC641CC11F4C8 \
  ; do \
    gpg --keyserver ha.pool.sks-keyservers.net --recv-keys "$key"; \
  done

ENV NODE_VERSION 4.4.5

RUN curl -SLO "https://nodejs.org/dist/v$NODE_VERSION/node-v$NODE_VERSION -linux-armv7l.tar.gz" \
  && curl -SLO "https://nodejs.org/dist/v$NODE_VERSION/SHASUMS256.txt.asc" \
  && gpg --batch --decrypt --output SHASUMS256.txt SHASUMS256.txt.asc \
  && grep " node-v$NODE_VERSION-linux-armv7l.tar.gz\$" SHASUMS256.txt | sha256sum -c - \
  && tar -xzf "node-v$NODE_VERSION-linux-armv7l.tar.gz" -C /usr/local --strip-components=1 \
  && rm "node-v$NODE_VERSION-linux-armv7l.tar.gz" SHASUMS256.txt.asc SHASUMS256.txt

COPY scripts/rpi_docker/entrypoint /entrypoint
RUN mkdir /aaaida
WORKDIR /aaaida
CMD ["/entrypoint"]

EXPOSE 40000

COPY . /aaaida
\end{lstlisting}

\section{Docker-compose.yml modificado}
\begin{lstlisting}[language=docker-compose-2,breaklines=true,label={code:compose}]
version: '2'
services:
 mongo:
  container_name: mongo
  restart: always
  image: partlab/ubuntu-arm-mongodb
  ports:
   - "27017:27017"
  volumes:
   - mongo-data:/data/db
  command: /usr/bin/mongod --smallfiles --journal
 aaaida:
  container_name: aaaidaArm
  restart: always
  network_mode: "home"
  privileged: true
  image: alteraid/aaaida-datastore-arm
  ports:
   - "40000:40000"
  environment:
   - NODE_ENV=docker
  volumes:
   - aaaida-data:/mnt/aaaidajs/

volumes:
 mongo-data:
   driver: local
 aaaida-data:
   driver: local  
\end{lstlisting}


%%%%%%%%%%%%%%%%%%%%%%%%%%%%%%%%%%%%%%%%%%%%%%%%%%%%%%%%%%%%%%%%%%%%%%%%%%
%%%%%%%%%%%%%%%%%%%%%%%%%%%%%%%%%%%%%%%%%%%%%%%%%%%%%%%%%%%%%%%%%%%%%%%%%%
%%%%%%%%%%%%%%%%%%%%%%%%%%%%%%%%%%%%%%%%%%%%%%%%%%%%%%%%%%%%%%%%%%%%%%%%%%
% i  aixo es tot! ;)
\end{document}







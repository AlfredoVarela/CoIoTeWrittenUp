\chapter{Emular la Raspberry Pi en Qemu}

\begin{verbatim}
----------Compilar qemu con soporte ARM 1176---------------

# apt-get install git zlib1g-dev libsdl1.2-dev
# apt-get install libpixman-1-dev libfdt-dev
$ mkdir raspberrypi
$ cd raspberrypi
$ git clone git://git.qemu-project.org/qemu.git
$ cd qemu
$ ./configure --target-list="arm-softmmu arm-linux-user" --enable-sdl 
--prefix=/usr
$ make
# make install

---------- Descargar la imagen de la raspberry pi y el kernel-qemu------

$ wget "http://downloads.raspberrypi.org/raspbian_latest"
$ unzip 2016-03-18-raspbian-jessie.img.zip    //(la versión que te descarges)
$ rm -rf 2016-03-18-raspbian-jessie.img.zip	  //(la versión descargada)

$ fdisk -l 2016-03-18-raspbian-jessie.img 
$ sudo mount -v -o offset=67108864 -t ext4 2016-03-18-raspbian-jessie.img /mnt/
$ cd /mnt
$ sudo nano ./etc/ld.so.preload  //(comentar el codigo)
$ sudo nano ./etc/fstab 	 	 //(Comentar la segunda linea)
$ cd ~
$ sudo umount /mnt

------------Ejecutar por primera vez-------------------------

$ qemu-system-arm -kernel kernel-qemu -cpu arm1176 -m 256 -M versatilepb 
-no-reboot -serial stdio -append "root=/dev/sda2 panic=1 rootfstype=ext4 rw" 
-hda 2016-03-18-raspbian-jessie.img -redir tcp:2222::22


------------ Resize de la imagen ---------------------

$ qemu-img resize raspi.img +16024M 
$ qemu-system-arm -M versatilepb -cpu arm1176 -hda 2016-03-18-raspbian-jessie.img 
-kernel kernel-qemu -m 192 -append "root=/dev/sda2" -drive file=raspi.img
$ sudo apt-get -y install gparted
$ sudo gparted
$ qemu-system-arm -kernel kernel-qemu -cpu arm1176 -m 256 -M versatilepb 
-no-reboot -serial stdio -append "root=/dev/sda2 panic=1 rootfstype=ext4 rw"
-hda raspi.img -redir tcp:2222::22
\end{verbatim}

Puede haber problemas con la interfaz gráfica, ya que si no hay suficiente memoria en la imagen, se deshabilita. 
